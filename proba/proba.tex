\documentclass[11pt,class=report,crop=false]{standalone}
\usepackage[screen]{../python}



\begin{document}


%====================================================================
\chapitre{Probabilities -- Parrondo's paradox}
%====================================================================

\objectifs{You will program two simple games. When you play these games, you are more likely to lose than to win. However, when you play both games at the same time, you have a better chance of winning than losing! It's a paradoxical situation.}

%%%%%%%%%%%%%%%%%%%%%%%%%%%%%%%%%%%%%%%%%%%%%%%%%%%%%%%%%%%%%%%%
%%%%%%%%%%%%%%%%%%%%%%%%%%%%%%%%%%%%%%%%%%%%%%%%%%%%%%%%%%%%%%%%

\begin{cours}[Random -- Winning -- Expected value]

\index{random@\ci{random}}

A player plays a game of chance: he throws a coin; depending on the result he wins or loses money. We cannot predict how much the player will win or lose for sure, but we will introduce a value that estimates how much the player can expect to win on average if he plays many times.

\begin{itemize}
  \item At the start the player's total winnings are zero: $g=0$.
  
  \item Each time he draws, he throws a coin. If he wins, he gets one dollar, if he loses he owes one dollar.
  
  \item In the games we study, the coin is not balanced (it is a little rigged). The player does not have as many chances to win as to lose.
  
  \item Draws are repeated a number of times $N$. After $N$ draws, the player's winnings (which can be positive or negative) are totaled.
  
  \item The \defi{expected value},\index{expected value} is the amount that the player can expect to win with each throw. The expected value is estimated by averaging the earnings of a large number of draws. In other words, we have the formula:
   $$\text{expected value} \  \simeq \  \frac{\text{winnings afters $N$ throws}}{N} \qquad \text{with large $N$}.$$ 
   
   For our games, the expected value will be a real number between $-1$ and $+1$.
  
  \item Examples:
  \begin{itemize}
    \item If the coin is well balanced ($50$ chances out of $100$ to win),
    then for a large number of $N$ draws, the player will win about as many times as he loses; his winnings will be close to $0$ and so the expected value will be close to $\frac{0}{N} = 0$. On average he earns $0$ dollar per draw.
    
    \item If the coin is rigged and the player wins all the time, then after $N$ draws, he has pocketed $N$ dollars. The expected value is therefore $\frac{N}{N}=1$.
    
    \item If the coin is rigged so that the player loses all the time, then after $N$ draws, his win is $-N$ dollars. The expected value is therefore $\frac{-N}{N}=-1$.
    
    \item An expected value of $-0.5$ means that on average the player loses $0.5$ dollar per draw. This is possible with an unbalanced coin that wins in only one out of four cases. (Check the calculation!) If the player plays $1000$ times, we can estimate that he will lose $500$ dollars ($-0.5 \times 1000 = -500$).
 \end{itemize}     
\end{itemize}

\end{cours}

%%%%%%%%%%%%%%%%%%%%%%%%%%%%%%%%%%%%%%%%%%%%%%%%%%%%%%%%%%%%%%%%
%%%%%%%%%%%%%%%%%%%%%%%%%%%%%%%%%%%%%%%%%%%%%%%%%%%%%%%%%%%%%%%%

%%%%%%%%%%%%%%%%%%%%%%%%%%%%%%%%%%%%%%%%%%%%%%%%%%%%%%%%%%%%%%%%
% Activity 1
%%%%%%%%%%%%%%%%%%%%%%%%%%%%%%%%%%%%%%%%%%%%%%%%%%%%%%%%%%%%%%%%


\begin{activite}[Game A: first losing game]

\objectifs{Goal: first, model a simple game, where, on average, the player loses.}

\textbf{Game A.} In this first game, we throw a slightly unbalanced coin: the player wins one dollar in $49$ cases out of $100$; he loses one dollar in $51$ cases out of $100$.

\begin{enumerate}
  \item \textbf{One draw.}
  Write a \ci{throw_game_A()} function that does not depend on any argument and that models a draw of the game A. For this purpose:
  \begin{itemize}
    \item Pick a random number $0 \le x < 1$ using the \ci{random()} function of the \ci{random} module.
    \item Returns $+1$ if $x$ is smaller than $0.49$; and $-1$ otherwise.
  \end{itemize}
  
  
  \item \textbf{Gain.} Write a \ci{gain_game_A(N)} function that models $N$ draws from game A and returns the total winnings of these draws. Of course, the result depends on the actual runs, it can vary from one time to another.
  
  \item \textbf{Expected value.} Write an \ci{expected_value_game_A(N)} function that returns an estimate of the expected value of game A according to the formula:
   $$\text{expected value} \  \simeq \  \frac{\text{winnings afters $N$ throws}}{N} \qquad \text{with large $N$}.$$   
   
  \item \textbf{Conclusion.}
  \begin{enumerate}
    \item Estimate the expected value by making at least one million draws.
    \item What does it mean that the expected value is negative?
    \item Deduct from the expected value how much I can expect to win (or lose) by playing $1000$ times in game A.
  \end{enumerate}

\end{enumerate}

\end{activite}
  
  

%%%%%%%%%%%%%%%%%%%%%%%%%%%%%%%%%%%%%%%%%%%%%%%%%%%%%%%%%%%%%%%%
% Activity 2
%%%%%%%%%%%%%%%%%%%%%%%%%%%%%%%%%%%%%%%%%%%%%%%%%%%%%%%%%%%%%%%%


\begin{activite}[Game B: second losing game]

\objectifs{Goal: model a second game that is a little more complicated where on average the player still loses.}

\textbf{Game B.} The second game is a little more complicated. At the beginning, the player starts with a zero win: $g=0$. Then, depending on this gain, he plays one of the following two subgames:
\begin{itemize}
  \item \textbf{Subgame B1.} If the gain $g$ is a multiple of $3$, then he throws a very disadvantageous coin: the player wins a dollar in only $9$ cases out of $100$ (so he loses a dollar in $91$ cases out of $100$).
  
  \item \textbf{Subgame B2.} If the gain $g$ is not a multiple of $3$, then he throws an advantageous coin: the player wins a dollar in $74$ cases out of $100$ (so he loses a dollar in $26$ cases out of $100$).
\end{itemize}

\begin{enumerate}
  \item \textbf{One draw.}
  Write a \ci{throw_game_B(g)} function that depends on the amount already acquired and models a draw from game B. You can use the \ci{g\%3 == 0} test to find out if $g$ is a multiple of $3$.
  
  \item \textbf{Gain.} Write a \ci{gain_game_B(N)} function that models $N$ draws playing game B (starting from a zero initial gain) and returns the total gain of these draws.
  
 
  
  \item \textbf{Expected value.} Write a function \ci{expected_value_game_B(N)} that returns an estimate of the expected value of the game B.
  
  \item \textbf{Conclusion.}
  \begin{enumerate}
    \item Estimate the expected value by making at least one million draws.
    \item How much can I expect to win or lose by playing $1000$ times in game B?
  \end{enumerate}

\end{enumerate}

\end{activite}



%%%%%%%%%%%%%%%%%%%%%%%%%%%%%%%%%%%%%%%%%%%%%%%%%%%%%%%%%%%%%%%%
% Activity 3
%%%%%%%%%%%%%%%%%%%%%%%%%%%%%%%%%%%%%%%%%%%%%%%%%%%%%%%%%%%%%%%%


\begin{activite}[Games A and B: a winning game!]

\objectifs{Goal: invent a new game where at each round you play either game A or game B; strangely enough, on average, this game makes you win! That's Parrondo's paradox.}

\textbf{Games AB.} In this third game, each round you play either game A or game B (the choice is made at random). 
At the beginning the player starts with a zero win: $g=0$. 
At each step, he chooses at random ($50\%$ of luck each):
\begin{itemize}
  \item to play game A once,
  \item or to play game B once; more precisely with subgame B1 or subgame B2 depending on the win already acquired $g$.
\end{itemize}

\begin{enumerate}
  \item \textbf{One draw.}
  Write a \ci{throw_game_AB(g)} function that depends on the amount already acquired and models a draw of the AB game.
  
  \item \textbf{Gain.} Write a \ci{gain_game_AB(N)} function that models $N$ draws from the AB game (starting from a zero initial gain) and returns the total gain of these draws.
  
  \item \textbf{Expected value.} Write an \ci{expected_value_game_AB(N)} function that returns an estimate of the expected value of the AB game.
  
  \item \textbf{Conclusion.}
  \begin{enumerate}
    \item Estimate the expected value by performing at least one million game turns.
    \item What can we say this time about the sign of the expected value?
    \item How much can I expect to win or lose by playing the AB game $1000$ times? Surprising, isn't it?
  \end{enumerate}

\end{enumerate}
\end{activite}


\emph{Reference: } \og{}Parrondo's paradox\fg{}, Hélène Davaux, La gazette des mathématiciens, July 2017.


\end{document}
