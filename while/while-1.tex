\documentclass[11pt,class=report,crop=false]{standalone}
\usepackage[screen]{../python}

\begin{document}


%====================================================================
\chapitre{Arithmetic -- While loop -- I}
%====================================================================

\objectifs{The activities in this sheet focus on arithmetic: long division, prime numbers \ldots{} This is an opportunity to use the \og{}while\fg{} loop intensively.}



%%%%%%%%%%%%%%%%%%%%%%%%%%%%%%%%%%%%%%%%%%%%%%%%%%%%%%%%%%%%%%%%
%%%%%%%%%%%%%%%%%%%%%%%%%%%%%%%%%%%%%%%%%%%%%%%%%%%%%%%%%%%%%%%%

\begin{cours}[Arithmetic]

\index{Euclidean division}
\index{quotient}
\index{remainder}
\index{modulo}
\index{\ci{//}}
\index{\ci{\%}}

Let us recall what Euclidean division is. Here is the division of $a$ by $b$, $a$ is a positive integer, $b$ is a strictly positive integer (with the example of $100$ divided by $7$) :

\myfigure{1}{
\tikzinput{fig-while-en}
}

We have the two fundamental properties that define $q$ and $r$:
$$a = b \times q  + r \qquad \text{ and } \qquad 0 \le r < b$$

For example, for the division of $a=100$ by $b=7$: we have the quotient $q=14$ and the remainder $r=2$ that verify $a = b \times q  + r$ because $100 = 7 \times 14 + 2$ and also $r<b$ because $2<7$.

\mybox{
With \Python{} :
\begin{itemize}
  \item \ci{a // b} \quad returns the quotient,
  \item \ci{a \% b} \quad returns the remainder.
\end{itemize}
}

It is easy to check that:
\mycenterline{$b$ is a divisor of $a$ if and only if $r=0$.}

\end{cours}


%%%%%%%%%%%%%%%%%%%%%%%%%%%%%%%%%%%%%%%%%%%%%%%%%%%%%%%%%%%%%%%%
% Activity 1
%%%%%%%%%%%%%%%%%%%%%%%%%%%%%%%%%%%%%%%%%%%%%%%%%%%%%%%%%%%%%%%%

\begin{activite}[Quotient, remainder, divisibility]

\objectifs{Goal: use the remainder to find out if one integer divides another.}


\begin{enumerate}
  \item Program a function named \ci{quotient_remainder(a,b)} that does the following tasks for two integers $a\ge0$ and $b>0$ :
  \begin{itemize}
    \item It displays the quotient $q$ of the Euclidean division of $a$ per $b$,
    \item it displays the remainder $r$ of this division,
    \item it displays \ci{True} if the remainder $r$ is positive or zero and strictly less than $b$, and \ci{False} otherwise,
    \item it displays \ci{True} if you have equality $a = bq+r$, and \ci{False} if not.
   \end{itemize}
    
Here is an example of what the call should display for \ci{quotient_remainder(100,7)}:
\begin{lstlisting}  
Division of a = 100 by b = 7
The quotient is q = 14
The remainder is r = 2
Check remainder: 0 <= r < b? True
Check equality: a = bq + r? True
\end{lstlisting}

\emph{Note.} You have to check without cheating that we have $0 \le r<b$ and $a=bq+r$, but of course it must always be true!
  
  
  \item Program a function called \ci{is_even(n)} that tests if the integer $n$ is even or not. The function should return \ci{True} or \ci{False}.
  
  \emph{Hints.}
  \begin{itemize}
    \item First possibility: calculate \ci{n \% 2}.
    \item Second possibility: calculate \ci{n \% 10} (which returns the digit of units).
    \item The smartest people will be able to write the function with only two lines (one for \ci{def...} and the other for \ci{return...}).
   \end{itemize}
   
  \item Program a function called \ci{is_divisible(a,b)} that tests if $b$ divides $a$. The function should return \ci{True} or \ci{False}.
  
\end{enumerate}   
     
\end{activite}



%%%%%%%%%%%%%%%%%%%%%%%%%%%%%%%%%%%%%%%%%%%%%%%%%%%%%%%%%%%%%%%%
%%%%%%%%%%%%%%%%%%%%%%%%%%%%%%%%%%%%%%%%%%%%%%%%%%%%%%%%%%%%%%%%
\begin{cours}[\og{}while\fg{} loop]
The \og{}while\fg{} loop executes instructions as long as a condition is true.
As soon as the condition becomes false, it proceeds to the next instructions.

\index{loop!while}
\index{while@\ci{while}}

\mybox{
\myfigure{0.7}{
  \tikzinput{fig-while-lesson}
} }


\begin{exemple}
\begin{minipage}{0.55\textwidth}
Here is a program that displays the countdown $10,9,8,\ldots3,2,1,0$.
As long as the condition $n \ge 0$ is true, we reduce $n$ by $1$. The last value displayed is $n=0$, because then $n=-1$ and the condition \og{}$n \ge 0$\fg{} becomes false so the loop stops.
\end{minipage}\qquad\qquad
\begin{minipage}{0.4\textwidth}
\begin{lstlisting}
n = 10
while n >= 0:
    print(n)
    n = n - 1
\end{lstlisting}
\end{minipage}

\medskip

This is summarized in the form of a table:
  $$
  \begin{array}{l}
  \text{Input: $n=10$}    \\
  \begin{array}{|c|c|c|}
  \hline  
  n & \text{\og{}$n\ge0$\fg{} ?} & \text{new value of } n \\
  \hline\hline 
  10 & \text{yes} & 9 \\
  9 & \text{yes} & 8 \\
  \ldots & \ldots & \ldots \\
  1 & \text{yes} & 0 \\
  0 & \text{yes} & -1 \\
  -1 & \text{no} &  \\ 
  \hline
  \end{array} \\
  \text{Display: $10,9,8,7,6,5,4,3,2,1,0$}  
  \end{array} 
  $$ 

\end{exemple}


\begin{exemple}
\begin{minipage}{0.55\textwidth}
This piece of code looks for the first power of $2$ greater than a given integer $n$.
The loop prints the values $2$, $4$, $8$, $16$,\ldots{} It stops as soon as the power of $2$ is higher or equal to $n$, so this program displays $128$.
\end{minipage}\qquad\qquad
\begin{minipage}{0.4\textwidth}
\begin{lstlisting}
n = 100
p = 1
while p < n:
    p = 2 * p
print(p)
\end{lstlisting}
\end{minipage}

  $$
  \begin{array}{l}
  \text{Inputs: $n=100$, $p=1$}    \\
  \begin{array}{|c|c|c|}
  \hline  
  p & \text{\og{}$p < n$\fg{} ?} & \text{new value of } p \\
  \hline\hline 
  1 & \text{yes} & 2 \\
  2 & \text{yes} & 4 \\
  4 & \text{yes} & 8 \\
  8 & \text{yes} & 16 \\  
  16 & \text{yes} & 32 \\
  32 & \text{yes} & 64 \\ 
  64 & \text{yes} & 128 \\   
  128 & \text{no} &  \\     
  \hline
  \end{array} \\
  \text{Display: $128$}  
  \end{array} 
  $$ 
  
\end{exemple}


\begin{exemple}
\begin{minipage}{0.55\textwidth}
For this last loop we have already prepared a function called \ci{is_even(n)} which returns \ci{True} if the integer $n$ is even and \ci{False} otherwise. The loop does this: as long as the integer $n$ is even, $n$ becomes $n/2$. This amounts to removing all factors $2$ from the integer $n$. As $n = 56 = 2\times 2 \times 2 \times 7$, this program displays $7$.
\end{minipage}\qquad\qquad
\begin{minipage}{0.4\textwidth}
\begin{lstlisting}
n = 56
while is_even(n) == True:
    n = n // 2
print(n)
\end{lstlisting}



\end{minipage}

  $$
  \begin{array}{l}
  \text{Input: $n=56$}    \\
  \begin{array}{|c|c|c|}
  \hline  
  n & \text{\og{}is $n$ even\fg{} ?} & \text{new value of } n \\
  \hline\hline 
  56 & \text{yes} & 28 \\
  28 & \text{yes} & 14 \\
  14 & \text{yes} & 7 \\
  7 & \text{no} &  \\ 
  \hline
  \end{array} \\
  \text{Display: $7$}  
  \end{array} 
  $$ 
\medskip

For the latter example, it is much more natural to start the loop with
\mycenterline{\ci{while is_even(n):}}

Indeed \ci{is_even(n)} is already a value \og{}True\fg{} or \og{}False\fg{}.
Therefore we're getting closer to the English sentence \og{}while $n$ is even...\fg{}

\end{exemple}


\textbf{Operation \og{}\ci{+=}\fg{}.}
\index{+=@\ci{+=}}
To increment a number you can use these two methods:
\mycenterline{\ci{nb = nb + 1} \qquad or \qquad \ci{nb += 1}}
The second writing is shorter but makes the program less readable.
\end{cours}



%%%%%%%%%%%%%%%%%%%%%%%%%%%%%%%%%%%%%%%%%%%%%%%%%%%%%%%%%%%%%%%%
% Activity 2
%%%%%%%%%%%%%%%%%%%%%%%%%%%%%%%%%%%%%%%%%%%%%%%%%%%%%%%%%%%%%%%%


\begin{activite}[Prime numbers]

\objectifs{Goal: test if an integer is (or not) a prime number.}

\index{divisor}
\index{first number}

\begin{enumerate}
  \item \textbf{Smallest divisor.}
  
  Program a function called \ci{smallest_divisor(n)} that returns, the smallest divisor $d\ge2$ of the integer $n\ge2$.
  
  For example \ci{smallest_divisor(91)} returns $7$, because $91 = 7 \times 13$.
  
  \medskip
  
  \textbf{Method.}
  \begin{itemize}
    \item We remind you that $d$ divides $n$ if and only if \ci{n \% d} is equal to $0$.
    \item It is a bad idea is to use a loop \og{}for $d$ ranging from $2$ to $n$\fg{}, since, if for example we know that $7$ is a divisor of $91$ it is useless to test if $8,9,10\ldots$ are also divisors because we have already found a smaller one.
    
    \item A good idea is to use a \og{}while\fg{} loop!
    The principle is: \og{}as long as I haven't got my divisor, I should keep looking for\fg{}. (And so, as soon as I find it, I stop looking.)
    
    \item In practice here are the main lines:
    \begin{itemize}
      \item Begin with $d=2$.
      \item As long as $d$ does not divide $n$ move on to the next candidate ($d$ becomes $d+1$).
      \item At the end $d$ is the smallest divisor of $n$ (in the worst case $d=n$).
     \end{itemize} 
  \end{itemize}
  
  
  \item \textbf{Prime numbers (1).}
  
  Slightly modify your function \ci{smallest_divisor(n)} to write your first  prime function \ci{is_prime_1(n)} which returns \og{}True\fg{} if $n$ is a prime number and \og{}False\fg{} otherwise.
  
  For example \ci{is_prime_1(13)} returns \ci{True}, \ci{is_prime_1(14)} returns \ci{False}.
  
  \item \textbf{Fermat numbers.}
  
  Pierre de Fermat ($\sim$1605--1665) thought that all integers of the form $F_n = 2^{(2^n)}+1$ were prime numbers. Indeed $F_0=3$, $F_1=5$ and $F_2=17$
  are prime numbers. If he had known \Python{} he would probably have changed his mind! Find the smallest integer $F_n$ which is not prime.
  
  \emph{Hint.} With \Python{} $b^c$ is written \ci{b ** c} and therefore
  $a^{(b^c)}$ is written \ci{a ** (b ** c)}.
  
  \bigskip
  
  \emph{We will improve our function which tests if a number is prime or not, it will allow us to test lots of numbers or very large numbers more quickly.}  
  
  \item \textbf{Prime numbers (2).}
  
   Enhance your previous function to become \ci{is_prime_2(n)}. It should not test all the divisors $d$ from $2$ to $n$, but only up to $\sqrt{n}$.
   
  \emph{Explanations.}
  \begin{itemize}
    \item For example, to test if $101$ is a prime number, just see if it divisible by $2,3,\ldots,10$. It is faster!
    
    \item This improvement is due to the following proposal: if an integer is not prime then it admits a divisor $d$ that verifies $2 \le d \le \sqrt{n}$.
    
    \item Instead of testing if $d \le \sqrt{n}$, it is easier to test if $d^2 \le n$.
   \end{itemize} 
  
  
  \item \textbf{Prime numbers (3).}  
  
  Improve your function to become \ci{is_prime_3(n)} using the following idea.  We test if $n$ is divisible by $d=2$, but from $d=3$, we just test the odd divisors (we test $d$, then $d+2$\ldots).
 
 \begin{itemize}
   \item For example to test if $n = 419$ is a prime number, we first test if $n$ is divisible by $d=2$, then $d=3$ and then $d=5$, $d=7$\ldots 
   \item This allows you to do about half less tests!
   \item Explanations: if an even number $d$ divides $n$, then we already know that $2$ divides $n$.   
   \end{itemize}
  
  
  \item \textbf{Calculation time.}
  
  Compare the calculation times of your different functions \ci{is_prime()} by repeating the call \ci{is_prime(97)}, for example, a million times. See the course below for more information on how to do this.
\end{enumerate}   
     
\end{activite}


%%%%%%%%%%%%%%%%%%%%%%%%%%%%%%%%%%%%%%%%%%%%%%%%%%%%%%%%%%%%%%%%
%%%%%%%%%%%%%%%%%%%%%%%%%%%%%%%%%%%%%%%%%%%%%%%%%%%%%%%%%%%%%%%%
\begin{cours}[Calculation time]

There are two ways to make programs run faster: a good way and a bad way. The bad way is to buy a more powerful computer.
The good method is to find a more efficient algorithm!

With \Python, it is easy to measure the execution time of a function in order to compare it with the execution time of another. Just use the module \ci{timeit}.

\index{module!timeit@\ci{timeit}}
\index{timeit@\ci{timeit}}
\index{calculation time}

Here is an example: we measure the computation time of two functions that have the same purpose, test if an integer $n$ is divisible by $7$.

\begin{lstlisting}  
# First function (not very clever)
def my_function_1(n):
    divis = False
    for k in range(n):
        if k*7 == n:
            divis = True
    return divis

# Second function (faster)
def my_function_2(n):
    if n % 7 == 0:
        return True
    else:
        return False

# Measurement of execution times
import timeit

print(timeit.timeit("my_function_1(1000)", 
    setup="from __main__ import my_function_1", 
    number=100000))
print(timeit.timeit("my_function_2(1000)", 
    setup="from __main__ import my_function_2", 
    number=100000))
\end{lstlisting} 


\textbf{Results.} \\
The result depends on the computer, but allows the comparison of the execution times of the two functions.
\begin{itemize}
  \item The measurement for the first function (called $100\,000$ times) returns $5$ seconds. The algorithm is not very clever. We're testing
  if $7\times 1 =n$, then test $7\times 2 = n$, $7\times 3 = n$\ldots
  
  \item The measurement for the second function returns $0.01$ second! We test if the remainder of $n$ divided by $7$ is $0$. The second method is therefore $500$ times faster than the first.
\end{itemize}  

\medskip

\textbf{Explanations.}
\begin{itemize}
  \item The module is named \ci{timeit}.
    
  \item The function \ci{timeit.timeit()} returns the execution time in seconds.
The function takes the following parameters: 
  \begin{itemize}
    \item a string for the call of the function to be tested (here we ask if $1000$ is divisible by $7$),
    \item an argument \ci{setup="..."} which indicates where to find this function,
    \item the number of times you have to repeat the call to the function (here \ci{number=100000}).
  \end{itemize}
  \item The number of repetitions must be large enough to avoid uncertainties.
\end{itemize} 

\end{cours}


%%%%%%%%%%%%%%%%%%%%%%%%%%%%%%%%%%%%%%%%%%%%%%%%%%%%%%%%%%%%%%%%
% Activity 3
%%%%%%%%%%%%%%%%%%%%%%%%%%%%%%%%%%%%%%%%%%%%%%%%%%%%%%%%%%%%%%%%

\begin{activite}[More prime numbers]

\objectifs{Goal: program more \og{}while\fg{} loops and study different kinds of prime numbers using your \ci{is_prime()} function.}

\begin{enumerate}
  \item Write a function \ci{prime_after(n)} that returns the first prime number $p$ greater than or equal to $n$. 
  
  For example, the first prime number after $n=60$ is $p=61$. 
  What is the first prime number after $n=100\,000$?
  
  \item Two prime numbers $p$ and $p+2$ are called \defi{twin prime numbers}. Write a function \ci{twin_prime_after(n)} that returns the first pair $p$, $p+2$ of twin prime numbers, with $p \ge n$.
  
  For example, the first pair of twin primes after $n=60$ is $p=71$ and $p+2=73$.   
  What is the first pair of twin primes after $n=100\,000$?
    
  \item An integer $p$ is a \defi{Germain prime number} if $p$ and $2p+1$ are prime numbers. Write a function \ci{germain_after(n)} that returns the pair $p$, $2p+1$ where $p$ is the first Germain prime number $p \ge n$.
  
  For example, the first Germain prime number after $n=60$ is $p=83$, with $2p+1=167$.   
  What is the first Germain prime number after $n=100\,000$?
  
\end{enumerate}   
     
\end{activite}


\end{document}
