\documentclass[11pt,class=report,crop=false]{standalone}
\usepackage[screen]{../python}

\begin{document}


%====================================================================
\chapitre{Arithmetic -- While loop -- II}
%====================================================================

\objectifs{Our study of numbers is further developed with the \og{}while\fg{} loop. For this chapter you will need the \ci{is_prime()} function you wrote in the
\og{}Arithmetic -- While loop -- I\fg{} part.}



%%%%%%%%%%%%%%%%%%%%%%%%%%%%%%%%%%%%%%%%%%%%%%%%%%%%%%%%%%%%%%%%
% Activity 4
%%%%%%%%%%%%%%%%%%%%%%%%%%%%%%%%%%%%%%%%%%%%%%%%%%%%%%%%%%%%%%%%


\begin{activite}[Goldbach's conjecture(s)]

\objectifs{Goal: study two Goldbach conjectures. A conjecture is a statement that you think is true but you not know how to prove it.}
\index{Goldbach}

\begin{enumerate}
  \item \textbf{Goldbach's good guess: } \emph{Every even integer greater than $4$ is the sum of two prime numbers.}

  For example $4 = 2+2$, $6=3+3$, $8=3+5$, $10=3+7$ (but also $10=5+5$), $12 = 5+7$,\ldots{}
  For $n=100$ there are $6$ solutions: $100=3+97=11+89=17+83=29+71=41+59=47+53$.
  
  No one can prove this conjecture, but you will see that there are good reasons to believe it is true.
  
  \begin{enumerate}
    \item Program a \ci{number_solutions_goldbach(n)} function which for a given even integer $n$, finds how many decompositions $n=p+q$ there are where $p$ and $q$ are prime numbers and $p\le q$.
    
    For example, for $n=8$, there is only one solution $8=3+5$, but for $n=10$ there are two solutions $10 = 3+7$ and $10=5+5$.
 
   \emph{Hints.} 
   \begin{itemize}
     \item It is therefore necessary to test all $p$ including $2$ and $n/2$;
     \item set $q = n - p$;
     \item we have a solution when $p \le q$ and $p$ and $q$ are both prime numbers.
   \end{itemize}
   
   \item Prove with the machine that the Goldbach conjecture is verified for all even integers $n$ between $4$ and $10\,000$.
    
  \end{enumerate}
  
  \item \textbf{Goldbach's bad guess: } \emph{Every odd integer $n$ can be written as
  $$n = p + 2k^2$$
where $p$ is a prime number and $k$ is an integer (possibly zero).}
  
  \begin{enumerate}
    \item Program an \ci{is_decomposition_goldbach(n)} function that returns \og{}True\fg{} when there is a decomposition of the form $n=p+2k^2$.
    
    \item Show that Goldbach's second guess is wrong! There are two integers smaller than $10\,000$ that do not have a decomposition of this form. Find them!
  \end{enumerate} 
\end{enumerate}   
     
\end{activite}



%%%%%%%%%%%%%%%%%%%%%%%%%%%%%%%%%%%%%%%%%%%%%%%%%%%%%%%%%%%%%%%%
% Activity 5
%%%%%%%%%%%%%%%%%%%%%%%%%%%%%%%%%%%%%%%%%%%%%%%%%%%%%%%%%%%%%%%%

\begin{activite}[Numbers with 4 or 8 divisors]

\objectifs{Goal: disprove a conjecture by doing a lot of calculations.}

\textbf{Conjecture: } \emph{Between $1$ and $N$, there are more integers that have exactly $4$ divisors than integers that have exactly $8$ divisors.}

You will see that this conjecture looks true for $N$ that are rather small, but you will show that this conjecture is false by finding a large $N$ that contradicts this statement.

\begin{enumerate}
  \item \textbf{Number of divisors.}
  
  Program a \ci{number_of_divisors(n)} function that returns the number of integers dividing $n$.
  
  For example: \ci{number_of_divisors(100)} returns $9$ because there are $9$ divisors of $n=100$:
  $$1,2,4,5,10,20,25,50,100$$
  
  \emph{Hints.}
  \begin{itemize}
    \item Don't forget $1$ and $n$ as divisors.
    \item Try to optimize your function because you will use it intensively: for example, there are no divisors strictly larger than $\frac n2$ (except $n$).    
   \end{itemize}
   
   
   \item \textbf{4 or 8 divisors.}
   
   Program a \ci{four_and_eight_divisors(Nmin,Nmax)} function that returns two numbers: (1) the number of integers $n$ with $N_{\text{min}} \le n < N_{\text{max}}$ that admit exactly $4$ divisors and (2) the number of integers $n$ with $N_{\text{min}} \le n < N_{\text{max}}$ that admit exactly $8$ divisors.
   
   For example \ci{four_and_eight_divisors(1,100)} returns \ci{(32,10)} because there are $32$ integers between $1$ and $99$ that admit $4$ divisors, but only $10$ integers that admit $8$.
   
   \item \textbf{Proof that the conjecture is false.}
   
   Check that for \og{}small\fg{} values of $N$ (up to $N = 10\,000$ for example) there are more integers with $4$ divisors than $8$. But check that for $N=300\,000$ this is no longer the case.
   
   \emph{Hints.} As there are many calculations, you can split them into slices (the slice of integers $1\le n < 50\,000$, then $50\,000 \le n < 100\,000$,...) and then add them up.
   This allows you to split your calculations between several computers.
   
\end{enumerate}   
     
\end{activite}



%%%%%%%%%%%%%%%%%%%%%%%%%%%%%%%%%%%%%%%%%%%%%%%%%%%%%%%%%%%%%%%%
% Activity 6
%%%%%%%%%%%%%%%%%%%%%%%%%%%%%%%%%%%%%%%%%%%%%%%%%%%%%%%%%%%%%%%%

\begin{activite}[121111\ldots{} is never prime?]

\objectifs{Goal: study a new false conjecture!}

We call $U_k$ the following integer:
$$U_k = 1\,2 \underbrace{1\,1\,1\,\ldots\,1\,1\,1}_{k \text{ occurrences of } 1}$$
formed by the digit $1$, then the digit $2$, then $k$ times the digit $1$.

For example $U_0 = 12$, $U_1 = 121$, $U_2 = 1211$, \ldots




\begin{enumerate}
  \item Write a function \ci{one_two_one(k)} that returns the integer $U_k$.
  
  \emph{Hint.} You can notice that starting with $U_0=12$, we have the relationship
  $U_{k+1} = 10 \cdot U_k + 1$. So you can start with \ci{u = 12} and repeat a number of times \ci{u  = 10*u + 1}.
  
  \item Check with your machine that $U_0,\ldots,U_{20}$ are not prime numbers. 
  
  \medskip
  
 \emph{You might think it's still the case, but it's not true. The integer $U_{136}$ is a prime number! Unfortunately it is too big to be verified with our algorithms. In the following point we will define what is an \emph{almost prime number} to be able to push the calculations further.}
  
  \item Program a function \ci{is_almost_prime(n,r)} that returns \og{}True\fg{} if the integer $n$ does not admit any divisor $d$ such that $1< d \le r$ (we assume $r<n$). 
  
  For example: $n = 143 = 11 \times 13$ and $r=10$, then \ci{is_almost_prime(n,r)} is
  \og{}True\fg{} because $n$ does not allow any divisor less than or equal to $10$. (But of course, $n$ is not a prime number.)
  
  \emph{Hint.} Adapt your \ci{is_prime(n)} function!
  
  \item Find all the integers $U_k$ with $0\le k \le 150$ which are almost prime for 
  $r = 1 \,000\,000$ (i.e. they are not divisible by any integer $d$ with $1<d \le 1\,000\,000$).
  
  \emph{Hint.} In the list you must find $U_{136}$ (which is a prime number) but also $U_{34}$ which is not prime but whose smallest divisor is
  $10\,149\,217\,781$.
\end{enumerate}   
     
\end{activite}


%%%%%%%%%%%%%%%%%%%%%%%%%%%%%%%%%%%%%%%%%%%%%%%%%%%%%%%%%%%%%%%%
% Activity 7
%%%%%%%%%%%%%%%%%%%%%%%%%%%%%%%%%%%%%%%%%%%%%%%%%%%%%%%%%%%%%%%%

\begin{activite}[Integer square root]

\objectifs{Goal: calculate the integer square root of an integer.}

\index{square root}

Let $n \ge 0$ be an integer. 
The \defi{integer square root of $n$} is the largest integer $r\ge0$ such as $r^2 \le  n$. 
Another definition is to say that the integer square root of $n$ is $\sqrt{n}$ rounded down to the nearest integer.

Examples:
\begin{itemize}
  \item $n=21$, then the integer square root of $n$ is $4$ (because $4^2 \le 21$, but $5^2 > 21$). In other words, $\sqrt{21} = 4.58\ldots$, and we round down to the nearest integer, so it is $4$.
  
  \item $n=36$, then the integer square root of $n$ is $6$ (because $6^2 \le 36$, but $7^2 > 36$). In other words, $\sqrt{36} = 6$ and the integer square root is of course also $6$.    
\end{itemize}

\begin{enumerate}
  \item Write a first function that calculates the integer square root of an integer $n$, first by calculating $\sqrt{n}$, then rounding down.
  
  \emph{Hints.}
  \begin{itemize}
    \item For this question only, you can use the \ci{math} module of \Python. 
    \item In this module \ci{sqrt()} returns the real square root.
    \item The \ci{floor()} function of the same module returns the number rounded down to the nearest integer.
  \end{itemize}
  
  \item Write a second function that calculates the integer square root of an integer $n$, but this time according to the following method:
  \begin{itemize}
    \item Start with $p=0$.
    \item As long as $p^2 \le n$, increment the value of $p$ by $1$.
  \end{itemize}
  Think carefully about what the returned value should be (beware of the offset!).
  
  \item Write a third function that still calculates the integer square root of an integer $n$ with the algorithm described below. This algorithm is called the Babylonian method (or Heron's method or Newton's method).
  
    \begin{algorithme}
  Input: a positive integer $n$

  Output: its integer square root

  \begin{itemize}
    \item Start with $a=1$ and $b=n$.
    
    \item As long as $|a-b| > 1$, repeat:
    \begin{itemize} 
     \item $a \leftarrow (a+b)//2$,
     \item $b \leftarrow n // a$. 
    \end{itemize}          
         
    \item Return the minimum between $a$ and $b$: this is the integer square root of $n$.
  \end{itemize} 
           
 \end{algorithme}
 
% $p//q$ for the integer division (i.e. the quotient of the Euclidean division of $p$ per $q$) was noted. This is simply obtained by ordering \Python{} : \ci{p//q}.
 
  We do not explain how this algorithm works, but it is one of the most effective methods to calculate square roots. The numbers $a$ and $b$ provide, during execution, an increasingly precise interval containing of $\sqrt{n}$. 
  
  Here is a table that details an example calculation for the integer square root of $n=1664$.

  \medskip
  
$$\begin{array}{|c|c|c|} 
\hline
\text{Step} & a & b \\ \hline\hline
i = 0  &  a =  1    &  b =  1664 \\
i = 1  &  a =  832  &  b =  2 \\
i = 2  &  a =  417  &  b =  3 \\
i = 3  &  a =  210  &  b =  7 \\
i = 4  &  a =  108  &  b =  15 \\
i = 5  &  a =  61   &  b =  27 \\
i = 6  &  a =  44   &  b =  37 \\
i = 7  &  a =  40   &  b =  41 \\ \hline
\end{array}$$

\medskip

In the last step, the difference between $a$ and $b$ is less than or equal to $1$, so the integer square root is $40$. We can verify that this is correct because: $40^2 = 1600 \le 1664 < 41^2 = 1681$.
    
\end{enumerate}  
  
  \smallskip
  
\textbf{Bonus.} Compare the execution speeds of the three methods using \ci{timeit()}. See the \og{}Functions\fg{} chapter.

\end{activite}



%%%%%%%%%%%%%%%%%%%%%%%%%%%%%%%%%%%%%%%%%%%%%%%%%%%%%%%%%%%%%%%%
%%%%%%%%%%%%%%%%%%%%%%%%%%%%%%%%%%%%%%%%%%%%%%%%%%%%%%%%%%%%%%%%
\begin{cours}[Exit a loop]

It is not always easy to find the right condition for a \og{}while\fg{} loop. \Python{} has a command to immediately exit a \og{}while\fg{} loop
or a \og{}for\fg{} loop: this is the \ci{break}\index{break@\ci{break}}\index{loop!quit} command.

Here are some examples that use this \ci{break} command. As it is rarely an elegant way to write your program, alternatives are also presented.

\begin{exemple}

Here are different codes for a countdown from $10$ to $0$.

\begin{minipage}{0.33\textwidth}
\begin{lstlisting}
# Countdown
n = 10
while True: #Infinite loop
    print(n)
    n = n - 1
    if n < 0:
        break #Immediate stop
\end{lstlisting}
\end{minipage}\qquad\qquad
\begin{minipage}{0.35\textwidth}
\begin{lstlisting}
# Better (with a flag)
n = 10
finished = False
while not finished:
    print(n)
    n = n - 1   
    if n < 0:
        finished = True
\end{lstlisting}
\end{minipage}
\begin{minipage}{0.2\textwidth}
\begin{lstlisting}
# Even better 
n = 10
while n >= 0:
    print(n)
    n = n - 1
\end{lstlisting}
\end{minipage}
\end{exemple}

\begin{exemple}

Here are programs that search for the integer square root of $777$, i.e. the largest integer $i$ that satisfies $i^2 \le 777$. In the script on the left, the search is limited to integers $i$ between $0$ and $99$.

\begin{minipage}{0.4\textwidth}
\begin{lstlisting}
# Integer square root
n = 777
for i in range(100):
    if i**2 > n:
        break
print(i-1)
\end{lstlisting}
\end{minipage}\qquad\qquad
\begin{minipage}{0.4\textwidth}
\begin{lstlisting}
# Better
n = 777
i = 0 
while i**2 <= n:
    i = i + 1
print(i-1) 
\end{lstlisting}
\end{minipage}
\end{exemple}

\begin{exemple}

Here are programs that calculate the real square roots of the elements in a list, unless of course the number is negative. The code on the left stops before the end of the list, while the code on the right handles the problem properly. \index{try@\ci{try/except}}

\begin{minipage}{0.4\textwidth}
\begin{lstlisting}
# Square root of the elements
# of a list
mylist = [3,7,0,10,-1,12]
for element in mylist:
    if element < 0:
        break
    print(sqrt(element))
\end{lstlisting}
\end{minipage}\qquad\quad
\begin{minipage}{0.4\textwidth}
\begin{lstlisting}
# Better with try/except
mylist = [3,7,0,10,-1,12]
for element in mylist:
    try: 
        print(sqrt(element))
    except:
        print("Warning, I don't know 
        how to compute the 
        square root of",element)
\end{lstlisting}
\end{minipage}
\end{exemple}


\end{cours}

\end{document}
