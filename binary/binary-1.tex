\documentclass[11pt,class=report,crop=false]{standalone}
\usepackage[screen]{../python}
\begin{document}



%====================================================================
\chapitre{Binary I}
%====================================================================

\objectifs{The computers transform all data into numbers and manipulate only those numbers. These numbers are stored in the form of lists of $0$'s and $1$'s. It's the binary numeral system of numbers. To better understand this binary numeral system, you must first need to understand the decimal numeral system better.}

\index{binary}


%%%%%%%%%%%%%%%%%%%%%%%%%%%%%%%%%%%%%%%%%%%%%%%%%%%%%%%%%%%%%%%%
%%%%%%%%%%%%%%%%%%%%%%%%%%%%%%%%%%%%%%%%%%%%%%%%%%%%%%%%%%%%%%%%

\begin{cours}[Base $10$ system]

\index{base 10}

We usually denote integers with the decimal numeral system (based on $10$). For example, $70\,685$ is 
${\color{blue}7} \times 10\,000
+{\color{blue}0} \times 1\,000
+{\color{blue}6} \times 100
+{\color{blue}8} \times 10 
+{\color{blue}5} \times 1$ : 
$$
\begin{array}{|c|c|c|c|c|}
  \hline
  {\color{blue}7} & {\color{blue}0} & {\color{blue}6} & {\color{blue}8} & {\color{blue}5} \\ 
  \hline\hline
  10000 & 1000 & 100 & 10 & 1 \\\hline
  10^4 & 10^3 & 10^2 & 10^1 & 10^0 \\
  \hline
\end{array}
$$   
(we can see that $5$ is the number of units, $8$ the number of tens, $6$ the number of hundreds\ldots). 

It is necessary to understand the powers of $10$. We denote $10 \times 10 \times \cdots \times 10$ (with $k$ factors) by $10^k$.

$$
\begin{array}{|c|c|c|c|c|c|c|c|}
  \hline
  {\color{blue}d_{p-1}} & {\color{blue}d_{p-2}}& {\color{blue}\cdots} & {\color{blue}d_i}& {\color{blue}\cdots} & {\color{blue}d_2} & {\color{blue}d_1} & {\color{blue}d_0} \\ 
  \hline\hline
  10^{p-1} & 10^{p-2} & \cdots & 10^i &\cdots & 10^2 & 10^1 & 10^0 \\
  \hline
\end{array}
$$   
  
We calculate the integer corresponding to the digits $[d_{p-1},d_{p-2}, \ldots,d_2,d_1,d_0]$ by the formula :
$$n = {\color{blue}d_{p-1}} \times 10^{p-1} + {\color{blue}d_{p-2}} \times 10^{p-2} + \cdots + {\color{blue}d_i} \times 10^{i} +  \cdots + {\color{blue}d_2} \times 10^2 + {\color{blue}d_1} \times 10^1 + {\color{blue}d_0} \times 10^0$$
\end{cours}

%%%%%%%%%%%%%%%%%%%%%%%%%%%%%%%%%%%%%%%%%%%%%%%%%%%%%%%%%%%%%%%%
% Activity 1
%%%%%%%%%%%%%%%%%%%%%%%%%%%%%%%%%%%%%%%%%%%%%%%%%%%%%%%%%%%%%%%%


\begin{activite}[From decimal system to integer]

\objectifs{Goal: from the decimal notation, find the integer.}

Write a function \ci{decimal_to_integer(list_decimal)} which takes a list representing the decimal notation and calculates the corresponding integer.
  
  \begin{fonction}[\ci{decimal_to_integer()}]
  Use: \ci{decimal_to_integer(list_decimal)} \\\
  Input: a list of numbers between $0$ and $9$ \\
  Output: the integer whose decimal notation is the list
  
  \medskip
  Example: if the input is \ci{[1,2,3,4]}, the output is \ci{1234}.
  \end{fonction} 
  
  \emph{Hints.}  It is necessary to sum up elements of type:
    $$d_i \times 10^{i} \quad \text{ for } 0 \le i < p$$
    where $p$ is the length of the list and $d_i$ is the digit at index $i$ \emph{counting from the end} (i.e. from right to left). To manage the fact that the index $i$ used for the power of $10$ does not correspond to the index in the list, there are two solutions:
  \begin{itemize}
    \item understand that $d_i = $ \ci{mylist}$[p-1-i]$ where \ci{mylist} is the list of $p$ digits,
    \item or start by reversing \ci{mylist}.    
   \end{itemize}  
  
\end{activite}

%%%%%%%%%%%%%%%%%%%%%%%%%%%%%%%%%%%%%%%%%%%%%%%%%%%%%%%%%%%%%%%%
%%%%%%%%%%%%%%%%%%%%%%%%%%%%%%%%%%%%%%%%%%%%%%%%%%%%%%%%%%%%%%%%


\begin{cours}[Binary]
\sauteligne
\begin{itemize}
  \item \textbf{Power of 2.}
    We denote $2^k$ for $2 \times 2 \times \cdots \times 2$ (with $k$ factors). For example, $2^3 = 2 \times 2 \times 2 = 8$. 
$$
\begin{array}{|c|c|c|c|c|c|c|c|}
  \hline
  2^7  & 2^6 & 2^5 & 2^4 & 2^3 & 2^2 & 2^1 & 2^0 \\
  \hline
  128 & 64 & 32 & 16 & 8 & 4 & 2 & 1 \\ 
  \hline
\end{array}
$$
\index{power of 2}

  \item \textbf{Binary system: an example.}
  
    Integers can be represented using a binary numeral system, i.e. a notation where the only numbers are $0$ or $1$. In binary, the digits are called \defi{bits}.\index{bits}
     For example, $1.0.1.1.0.0.1$ (pronounce the numbers one by one) is the binary numeral system of the integer $89$. How to do you do this calculation? The same way we did using the base $10$, but instead using the powers of $2$.
$$
\begin{array}{|c|c|c|c|c|c|c|}
  \hline
  {\color{red}1} & {\color{red}0} & {\color{red}1} & {\color{red}1} & {\color{red}0} & {\color{red}0} & {\color{red}1} \\ 
  \hline\hline
  64 & 32 & 16  & 8 & 4 & 2 & 1 \\\hline
  2^6 & 2^5 & 2^4 & 2^3 & 2^2 & 2^1 & 2^0 \\
  \hline
\end{array}
$$


    So the system {\color{red}1}.{\color{red}0}.{\color{red}1}.{\color{red}1}.{\color{red}0}.{\color{red}0}.{\color{red}1} represents the integer: 
    $${\color{red}1} \times 64  + {\color{red}0} \times 32 + {\color{red}1} \times 16  + {\color{red}1} \times 8 + {\color{red}0} \times 4 + {\color{red}0} \times 2 + {\color{red}1} \times 1 = 64 + 16 + 8 + 1 = 89.$$
    
    
   \item \textbf{Binary system: formula.} 
 
 
  $$
\begin{array}{|c|c|c|c|c|c|c|c|}
  \hline
  {\color{red}b_{p-1}} & {\color{red}b_{p-2}}& {\color{red}\cdots} & {\color{red}b_i}& {\color{red}\cdots} & {\color{red}b_2} & {\color{red}b_1} & {\color{red}b_0} \\ 
  \hline\hline
  2^{p-1} & 2^{p-2} & \cdots & 2^i &\cdots & 2^2 & 2^1 & 2^0 \\
  \hline
\end{array}
$$   
  
We can calculate the integer corresponding to the bits $[b_{p-1},b_{p-2}, \ldots,b_2,b_1,b_0]$ as a sum of terms $b_i \times 2^i$, by the formula :
$$n = {\color{red}b_{p-1}} \times 2^{p-1} + {\color{red}b_{p-2}} \times 2^{p-2} + \cdots + {\color{red}b_i} \times 2^i +  \cdots + {\color{red}b_2} \times 2^2 + {\color{red}b_1} \times 2^1 + {\color{red}b_0} \times 2^0$$
  
    \item \textbf{\Python{} and the binary.} \Python{} accepts the binary decimal system directly as long as we use the prefix \og{}\ci{0b}\fg{}.
    Examples:
    \begin{itemize}
      \item with \ci{x = 0b11010}, then \ci{print(x)} displays \ci{26},
      \item with \ci{y = 0b11111}, then \ci{print(y)} displays \ci{31},
      \item and \ci{print(x+y)} displays \ci{57}.
    \end{itemize} 
 \end{itemize}     

\end{cours}


%%%%%%%%%%%%%%%%%%%%%%%%%%%%%%%%%%%%%%%%%%%%%%%%%%%%%%%%%%%%%%%%
% Activity 2
%%%%%%%%%%%%%%%%%%%%%%%%%%%%%%%%%%%%%%%%%%%%%%%%%%%%%%%%%%%%%%%%


\begin{activite}[From binary numeral system to integer]

\objectifs{Goal: convert binary values to integers.}

\begin{enumerate}

  \item Calculate integers whose binary numeral system is given below. You can do it by hand or get help from \Python. For example $1.0.0.1.1$ is equal to $2^4+2^1+2^0 = 19$ as confirmed by the command \ci{0b10011} which returns $19$.
  
  \begin{itemize}
    \item $1.1$, $1.0.1$, $1.0.0.1$, $1.1.1.1$
    \item $1.0.0.0.0$, $1.0.1.0.1$, $1.1.1.1.1$
    \item $1.0.1.1.0.0$, $1.0.0.0.1.1$
    \item $1.1.1.0.0.1.0.1$
  \end{itemize}
  

  \item Write a function \ci{binary_to_integer(list_binary)} which takes in a list representing a binary system value and calculates the corresponding integer.
  
  \begin{fonction}[\ci{binary_to_integer()}]
  Use: \ci{binary_to_integer(list_binary)} \\\
  Input: a list of bits, $0$ and $1$ \\
  Output: the integer whose binary numeral system is the list
  
  \medskip
  Examples: 
  \begin{itemize}
    \item input \ci{[1,1,0]}, output \ci{6}
    \item input \ci{[1,1,0,1,1,1]}, output \ci{55}
    \item input \ci{[1,1,0,1,0,0,1,1,0,1,1,1]}, output \ci{3383}
  \end{itemize}       
  \end{fonction}
  
  \emph{Hints.} This time it is necessary to sum up elements of type:
    $$b_i \times 2^{i} \quad \text{ for } 0 \le i < p$$
    where $p$ is the length of the list and $b_i$ is the bit ($0$ or $1$) at index $i$ of the list counting from the end.
  
  
  \item Here is an algorithm that does the same conversion: it allows the calculation of the integer from the binary notation, but it has the advantage of never directly calculating powers of $2$. Program this algorithm into a \ci{binary_to_integer_bis()} function that has the same characteristics as the previous function.
  
  \begin{algorithme}
  Input: \ci{mylist} : a list of $1$'s and $0$'s 
  
  Output: the associated binary number 

  \begin{itemize}
    \item Initialize a variable $n$ to $0$.

    
    \item For each element $b$ from \ci{mylist} :
    
     \begin{itemize} 
       \item if $b$ is $0$, then make $n \leftarrow 2n$,
       \item if $b$ is $1$, then make $n \leftarrow 2n+1$.
     \end{itemize}    
         
    \item The result is the value of $n$.
  \end{itemize} 
             
 \end{algorithme}
 
\end{enumerate}

\end{activite}
  
  
%%%%%%%%%%%%%%%%%%%%%%%%%%%%%%%%%%%%%%%%%%%%%%%%%%%%%%%%%%%%%%%%
%%%%%%%%%%%%%%%%%%%%%%%%%%%%%%%%%%%%%%%%%%%%%%%%%%%%%%%%%%%%%%%%

\begin{cours}[Base $10$ system (again)]

Of course, when you see the number $1234$ you can immediately find the list of its digits $[1,2,3,4]$. But how to do it in general from an integer $n$?
 
\begin{enumerate}
  \item We will need to use the Euclidean division by $10$:
  \begin{itemize}
    \item calculate the \defi{remainder} left over after dividing by $10$ using \ci{n \% 10}, we also call this \defi{$n$ modulo $10$}
    \item calculate \ci{n // 10} or the \defi{quotient} of this division. 
  \end{itemize}
  
\myfigure{1}{
\tikzinput{fig-binary-1}
}
  
  \item The \Python{} commands are simply \ci{n \% 10} and \ci{n // 10}.
  
  Example: \ci{1234 \% 10} is equal to \ci{4}; \ci{1234 // 10} is equal to \ci{123}.
  
  \item
  \begin{itemize}
    \item The digit of the units of $n$ is obtained using modulo $10$: it is $n\%10$. For example $1234 \% 10 = 4$.
    \item The tens figure is obtained from the quotient of $n$ 
    divided by $10$ and then taking modulo $10$: it is therefore $(n // 10) \% 10$.
   Example: $1234 // 10 = 123$, then we have $123 \% 10 = 3$; the figure of tens of $1234$ is indeed $3$.
   
   \item For the figure of hundreds, we divide $n$ by $100$, then take modulo $10$.  Example $1234 // 100 = 12$; $2$ is the figure of the units of $12$ and it is the figure of the hundreds of $1234$. Note: dividing by $100$ means dividing by $10$, then again by $10$.
   
   \item For the figure of thousands we calculate the quotient of the division by $1000$ then we take the figure of the units\ldots
   \end{itemize} 
 \end{enumerate}     

\end{cours}



%%%%%%%%%%%%%%%%%%%%%%%%%%%%%%%%%%%%%%%%%%%%%%%%%%%%%%%%%%%%%%%%
% Activity 3
%%%%%%%%%%%%%%%%%%%%%%%%%%%%%%%%%%%%%%%%%%%%%%%%%%%%%%%%%%%%%%%%


\begin{activite}[Find the digits of an integer]

\objectifs{Goal: decompose an integer into a list of its digits (based on $10$).}

Program the function \ci{integer_to_decimal()} using the following algorithm.

 
   \begin{fonction}[\ci{integer_to_decimal()}]
  Use: \ci{integer_to_decimal(n)} \\
  Input: a positive integer \\
  Output: the list of its digits
  
  \medskip
    
  Example: if the input is \ci{1234}, the output is \ci{[1,2,3,4]}.
  \end{fonction}


  \begin{algorithme}
  Input: an integer $n>0$

  Output: the list of its digits 

  \begin{itemize}
    \item Start from an empty list.
    
    \item As long as $n$ is not zero:
    
     \begin{itemize} 
       \item add $n \% 10$ at the beginning of the list,
       \item set $n \leftarrow n//10$.
     \end{itemize}    
         
    \item The result is the desired list.
  \end{itemize} 
             
 \end{algorithme}
 
  

\end{activite}
%%%%%%%%%%%%%%%%%%%%%%%%%%%%%%%%%%%%%%%%%%%%%%%%%%%%%%%%%%%%%%%%
%%%%%%%%%%%%%%%%%%%%%%%%%%%%%%%%%%%%%%%%%%%%%%%%%%%%%%%%%%%%%%%%



\begin{cours}[Binary system with \Python]

\Python{} calculates the binary value of an integer very well using the \ci{bin()} function.
\index{bin@\ci{bin}}
  
   \begin{fonctionpython}[\ci{python : bin()}]
    Use: \ci{bin(n)}\\
    Input: an integer \\
    Output: the binary numeral system of $n$ in the form of a string starting with \ci{'0b'}
  
  \medskip
     
   Example:
  \begin{itemize}  
    \item \ci{bin(37)} returns \ci{'0b100101'}
    \item \ci{bin(139)} returns \ci{'0b10001011'}
  \end{itemize} 
  \end{fonctionpython}  
   
\end{cours}


\begin{cours}[Binary system calculation]

Calculating the binary value of an integer $n$, is the same as calculating the decimal value except replace the divisions by $10$ with a divisions by $2$.

So we need:
  \begin{itemize}
    \item $n\%2$ : the remainder of the Euclidean division of $n$ by $2$ (also called $n$ modulo $2$); the remainder is either $0$ or $1$.
    \item and $n//2$: the quotient of this division. 
  \end{itemize}
  
  Note that the remainder $n\%2$ is either $0$ (when $n$ is even) or $1$ (when $n$ is odd).
  
  Here is a general method to calculate the binary value of an integer:
\begin{itemize}
  \item We start from the integer whose binary value we want.
  
  \item A series of Euclidean divisions by 2 are performed: 
  \begin{itemize}
    \item for each division, you get a remainder 0 or 1; 
    \item we get a quotient that we divide again by 2\ldots{} We stop when this quotient is zero.
  \end{itemize}
  
  \item The binary system is read as the sequence of the remainders, but starting from the last one.
\end{itemize}

\begin{exemple}
Calculation of the binary value of 14.

\begin{itemize}
  \item We divide 14 by 2, the quotient is 7, the remainder is 0.
  \item We divide 7 (the previous quotient) by 2: the new quotient is 3, the new remainder is 1.
  \item We divide 3 by 2: quotient 1, remainder 1.
  \item We divide 1 by 2: quotient 0, remainder 1.
  \item It's over (the last quotient is zero).
  \item The successive remainders are 0, 1, 1, 1. we read the binary numeral system backwards, it is 1.1.1.0.  
\end{itemize}

The divisions are done from left to right, but the remainders are read from right to left.

\myfigure{1}{
\tikzinput{fig-binary-2}
}
\end{exemple}

\begin{exemple}
Binary value of 50.

\myfigure{1}{
\tikzinput{fig-binary-3}
}

The successive remainders are 0, 1, 0, 0, 1, 1, so the binary notation of 50 is 1.1.0.0.1.0.
\end{exemple}


\end{cours}



%%%%%%%%%%%%%%%%%%%%%%%%%%%%%%%%%%%%%%%%%%%%%%%%%%%%%%%%%%%%%%%%
% Activity 4
%%%%%%%%%%%%%%%%%%%%%%%%%%%%%%%%%%%%%%%%%%%%%%%%%%%%%%%%%%%%%%%%


\begin{activite}[]

\objectifs{Goal: find the binary value of an integer.}

\begin{enumerate}
  \item Calculate the binary value of the following integers by hand. Check your results using the \Python{} \ci{bin()} function.
  
   \begin{itemize}
    \item $13$, $18$, $29$, $31$,
    \item $44$, $48$, $63$, $64$,
    \item $100$, $135$, $239$, $1023$.
  \end{itemize} 
  
  \item
  Program the \ci{integer_to_binary()} function using the following algorithm.

  \begin{algorithme}
  Input: an integer $n>0$

  Output: its binary value in the form of a list

  \begin{itemize}
    \item Start from an empty list.
    
    \item As long as $n$ is not zero:
    
     \begin{itemize} 
       \item add $n \% 2$ at the beginning of the list,
       \item set $n \leftarrow n//2$.
     \end{itemize}    
         
    \item The result is the desired list.
  \end{itemize} 
             
 \end{algorithme}
 
   \begin{fonction}[\ci{integer_to_binary()}]
  Use: \ci{integer_to_binary(n)} \\
  Input: a positive integer \\
  Output: its binary notation in the form of a list
  
  \medskip  
  Example: if the input is \ci{204}, the output is \ci{[1,1,0,0,1,1,0,0]}.
  \end{fonction} 
  
  Make sure your functions are working properly:
  \begin{itemize}
    \item start with any integer $n$,
    \item calculate its binary value,
    \item calculate the integer associated with this binary notation,
    \item you should find the starting integer!
  \end{itemize}
\end{enumerate}
\end{activite}

\end{document}
