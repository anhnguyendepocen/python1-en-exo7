\documentclass[11pt,class=report,crop=false]{standalone}
\usepackage[screen]{../python}

\begin{document}

%====================================================================
\chapitre{Binary II}
%====================================================================

\objectifs{We continue our exploration of the world of $0$ and $1$.}

%%%%%%%%%%%%%%%%%%%%%%%%%%%%%%%%%%%%%%%%%%%%%%%%%%%%%%%%%%%%%%%%
% Activity 1
%%%%%%%%%%%%%%%%%%%%%%%%%%%%%%%%%%%%%%%%%%%%%%%%%%%%%%%%%%%%%%%%

\index{binary}


\begin{activite}[Palindromes]

\objectifs{Goal: find palindromes in binary and decimal numeral system.}
\index{palindrome}

In English a palindrome is a word (or a sentence) that can be read in both directions, for example \og{}\mot{RADAR}\fg{} or \og{}\mot{A MAN, A PLAN, A CANAL: PANAMA}\fg{}.
In this activity, a \defi{palindrome} will be a list, which has the same elements when browsing from left to right or right to left.

Examples:
\begin{itemize}
  \item \ci{[1,0,1,0,1]} is a palindrome (with binary numeral system),
  \item \ci{[2,9,4,4,9,2]} is a palindrome (with decimal numeral system).
\end{itemize}

\begin{enumerate}

  \item Program a function \ci{is_palindrome(mylist)} that tests if a list is a palindrome or not. 
  
  \emph{Hints.} You can compare the items at ranks $i$ and $p-1-i$ or use \ci{list(reversed(liste))}.
 
  \item We are looking for integers $n$ such that their binary numeral system is a palindrome. For example, the binary notation of $n=27$ is the palindrome \ci{[1,1,0,1,1]}. This is the tenth integer $n$ having this property. 
  
  What is the thousandth integer $n\ge0$ whose binary notation is a palindrome?
  
  \item What is the thousandth integer $n\ge0$ whose decimal notation is a palindrome?
  
  For example, the digits of $n=909$ in decimal notation, form the palindrome \ci{[9,0,9]}. This is the hundredth integer $n$ having this property.
  
  \item An integer $n$ is a \defi{bi-palindrome} if its binary notation \emph{and} its decimal notation are palindromes. For example $n=585$ has a decimal notation which is a palindrome and also is its binary notation 
  \ci{[1,0,0,1,0,0,1,0,0,1]}. This is the tenth integer $n$ having this property.
  
  What is the twentieth integer $n\ge0$ to be a bi-palindrome?
  
\end{enumerate}

\end{activite}
  
   

%%%%%%%%%%%%%%%%%%%%%%%%%%%%%%%%%%%%%%%%%%%%%%%%%%%%%%%%%%%%%%%%
% Activity 2
%%%%%%%%%%%%%%%%%%%%%%%%%%%%%%%%%%%%%%%%%%%%%%%%%%%%%%%%%%%%%%%%

\begin{cours}[Logical operations]

\index{logic operation}

We consider that $0$ represents \og{}False\fg{} and $1$ is \og{}True\fg{}. 
\begin{itemize}
  \item With the logical operation \og{}OR\fg{}, the result is true as soon as at least one of the two terms is true. This is written:
  \begin{itemize}
    \item 0 OR 0 = 0
    \item 0 OR 1 = 1
    \item 1 OR 0 = 1
    \item 1 OR 1 = 1
   \end{itemize}
   
  \item With the logical operation \og{}AND\fg{}, the result is true only when both terms are true. This is written:
  \begin{itemize}
    \item 0 AND 0 = 0
    \item 0 AND 1 = 0
    \item 1 AND 0 = 0
    \item 1 AND 1 = 1
   \end{itemize}  
   
  \item The logical operation \og{}NOT\fg{}, exchange true and false values:
  \begin{itemize}
    \item NOT 0 = 1
    \item NOT 1 = 0
   \end{itemize}
   
    \item For numbers in binary notation, these operations range from bits to bits, i.e. digit by digit (starting with the digits on the right) as one would add (without carry). 
    
 For example: 
    
 \myfigure{1}{
\tikzinput{fig-binary-4}
}     

If the two systems do not have the same number of bits, we add non-significant $0$ on the left (example of $1.0.0.1.0$ OR $1.1.0$ on the figure at the right).
\end{itemize}  
  
   

\end{cours}


\begin{activite}[Logical operations]

\objectifs{Goal: program the main logical operations.}

\begin{enumerate}
  \item 
  \begin{enumerate}
    \item Program a function \ci{NOT()}\index{not@\ci{not}} which corresponds to the negation for a given list. For example, \ci{NOT([1,1,0,1])} returns \ci{[0,0,1,0]}.
    \item Program a function \ci{OReq()}\index{or@\ci{or}} which corresponds to \og{}OR\fg{} with two lists of equal length. For example, with \ci{mylist1 = [1,0,1,0,1,0,1]} and \ci{mylist2 = [1,0,0,1,0,0,1]}, the function returns \ci{[1,0,1,1,1,0,1]}.
    \item Do the same work with \ci{ANDeq()}\index{and@\ci{and}} for two lists having the same length.
  \end{enumerate}
  
  \item Write a function \ci{zero_padding(mylist,p)} that adds zeros at the beginning of the list to get a list of length \ci{p}.
  Example: if \ci{mylist = [1,0,1,1]} and \ci{p = 8}, then the function returns \ci{[0,0,0,0,1,0,1,1]}.
  
  \item Write two functions \ci{OR()} and \ci{AND()} which correspond to the logical operations, but with two lists that do not necessarily have the same length. 
  
  Example:
  \begin{itemize}
    \item \ci{mylist1 = [1,1,1,0,1]} and \ci{mylist2 = [1,1,0]},
    \item it should be considered that \ci{mylist2} is equivalent to the list 
    \ci{mylist2bis = [0,0,1,1,0]} of the same length as \ci{mylist1},
    \item so \ci{OR(mylist1,mylist2)} returns \ci{[1,1,1,1,1]},
    \item then \ci{AND(mylist1,mylist2)} returns \ci{[0,0,1,0,0]} (or \ci{[1,0,0]} depending on your choice).
\end{itemize}  
  
  \emph{Hints.} You can take over the content of your functions \ci{OReq} and \ci{ANDeq}, or you can first add zeros to the shortest list.
\end{enumerate}

\end{activite}


  

%%%%%%%%%%%%%%%%%%%%%%%%%%%%%%%%%%%%%%%%%%%%%%%%%%%%%%%%%%%%%%%%
% Activity 3
%%%%%%%%%%%%%%%%%%%%%%%%%%%%%%%%%%%%%%%%%%%%%%%%%%%%%%%%%%%%%%%%


\begin{activite}[De Morgan's laws]

\objectifs{Goal: generate all possible lists of $0$ and $1$ to check a proposition.}
\index{Morgan's laws}

\begin{enumerate}
  \item \textbf{First method: use binary notation.}
  
  We want to generate all possible lists of $0$ and $1$ of a given size $p$. Here's how to do it:
  \begin{itemize}
    \item An integer $n$ runs all integers from $0$ to $2^p -1$.
    \item For each of these integers $n$, we calculate its binary notation (in the form of a list).
    \item We add (if necessary) $0$ at the beginning of the list, in order to get a list of length $p$.
  \end{itemize}
  
  Program this method.
  
  Example: for $n = 36$, its binary notation is \ci{[1,0,0,1,0,0]}. If you want a list of $p=8$ bits, you add two $0$:
   \ci{[0,0,1,0,0,1,0,0]}.
  
  
  \item \textbf{Second method (optional): a recursive algorithm.}
  
  We want to generate again all the possible lists of $0$ and $1$ of a given size. We adopt the following procedure: if we know how to find all the lists of size $p-1$, then to obtain all the lists of size $p$, we just have to add one $0$ at the beginning of each list of size $p-1$, then to start again by adding one $1$ at the beginning of each list of size $p-1$. 
  
  For example, there are $4$ lists of length $2$: \ci{[0, 0]}, \ci{[0, 1]}, \ci{[1, 0]}, \ci{[1, 1]}. I deduct the $8$ lists of length $3$:
  \begin{itemize}
    \item $4$ lists by adding $0$ at the front: \ci{[0, 0, 0]}, \ci{[0, 0, 1]}, \ci{[0, 1, 0]}, \ci{[0, 1, 1]}, 
    \item $4$ lists by adding $1$ at the front: \ci{[1, 0, 0]}, \ci{[1, 0, 1]}, \ci{[1, 1, 0]}, \ci{[1, 1, 1]}.
  \end{itemize}
  
  This gives the following algorithm, which is a recursive algorithm (because the function calls itself).
  \begin{algorithme}
  Use: \ci{every_binary_number(p)}\\
  Input: an integer \ci{p}\,$>0$\\
  Output: the list of all possible lists of $0$ and $1$ of length \ci{p}

  \begin{itemize}
% \item If \ci{p}$=0$ send back the empty list \ci{[]}.  
    \item If \ci{p}\,$=1$ return the list \ci{[ [0], [1] ]}.     
    \item If \ci{p}\,$\ge2$, then:
    \begin{itemize}
      \item get all lists of size \ci{p-1} by the call \ci{every_binary_number(p-1)}
      \item for each item in this list, build two new items:
       \begin{itemize} 
         \item on the one hand add \ci{0} at the beginning of this element;
         \item on the other hand add \ci{1} at the beginning of this element;
         \item then add these two items to the list of lists of size \ci{p}.
       \end{itemize}      
    \end{itemize}
    
    \item Return the list of all the lists with a size \ci{p}.
  \end{itemize}     
 \end{algorithme}  
  
    \item \textbf{De Morgan's laws.} 
    
    De Morgan's laws state that for booleans (true/false) or bits (1/0), we always have these properties:
    $$\text{NOT}( b_1 \text{ OR } b_2 ) = 
    \text{NOT}( b_1 )  \text{ AND } \text{NOT}(b_2)
    \qquad
    \text{NOT}( b_1 \text{ AND } b_2 ) = 
    \text{NOT}( b_1 )  \text{ OR } \text{NOT}(b_2).$$
    
    Experimentally checks that these equations are still true for any list $\ell_1$ and $\ell_2$ of exactly $8$ bits.
    
\end{enumerate}

\end{activite}

\end{document}
